\documentclass[12pt,a4paper,spanish]{report}
\usepackage[utf8]{inputenc}
\usepackage[T1]{fontenc}
\usepackage[spanish,es-tabla]{babel}

% --- CORRECCIÓN 1: Márgenes más ajustados arriba ---
\usepackage[top=2cm, bottom=2.5cm, left=2.5cm, right=2.5cm]{geometry}

\usepackage{graphicx}
\usepackage{xcolor}
\usepackage{listings}
\usepackage{fancyhdr}
\usepackage{float}
\usepackage{array}
\usepackage{booktabs}
\usepackage{hyperref}
\usepackage{titlesec}
\usepackage{tocloft}

% Configuración de colores
\definecolor{darkblue}{rgb}{0,0,0.5}
\definecolor{codecolor}{rgb}{0.95,0.95,0.95}
\definecolor{keywordcolor}{rgb}{0.1,0.1,0.7}
\definecolor{stringcolor}{rgb}{0.6,0.1,0.1}

% Estilos de código
\lstset{
    language=bash,
    backgroundcolor=\color{codecolor},
    basicstyle=\ttfamily\footnotesize,
    breaklines=true,
    captionpos=b,
    commentstyle=\color{gray},
    keywordstyle=\color{keywordcolor},
    stringstyle=\color{stringcolor},
    frame=single,
    frameround=tttt,
    rulecolor=\color{darkblue},
    showstringspaces=false,
    tabsize=2,
    inputencoding=utf8,
    extendedchars=true,
    literate={á}{{\'a}}1 {é}{{\'e}}1 {í}{{\'i}}1 {ó}{{\'o}}1 {ú}{{\'u}}1 {ñ}{{\~n}}1
}

% Configuración de encabezados y pies
\pagestyle{fancy}
\fancyhf{}
\setlength{\headheight}{15pt} % Evita advertencias de altura
\fancyhead[L]{\textit{Configuración de Red IPv6}}
\fancyhead[R]{\thepage}
\fancyfoot[C]{\small ITSOEH -- Configuración de Redes IPv6}

% --- CORRECCIÓN 2: Títulos pegados arriba ---

% Formato visual del capítulo
\titleformat{\chapter}[hang]
{\normalfont\huge\bfseries\color{darkblue}}
{\thechapter.}{20pt}{\huge}

% *** ESTA LÍNEA ES LA CLAVE ***
% El -50pt elimina el espacio blanco nativo de LaTeX antes del título
\titlespacing*{\chapter}{0pt}{-50pt}{20pt}

% Formato de secciones
\titleformat{\section}[hang]
{\normalfont\Large\bfseries\color{darkblue}}
{\thesection.}{12pt}{\Large}

\titleformat{\subsection}[hang]
{\normalfont\large\bfseries}
{\thesubsection.}{10pt}{\large}

% Referencias
\hypersetup{
    colorlinks=true,
    linkcolor=darkblue,
    urlcolor=darkblue,
    citecolor=darkblue
}

% Datos de portada
\title{\huge\textbf{REPORTE TÉCNICO}\\\large Configuración IPv6}
\author{[Nombre]}
\date{\today}

% ... (Aquí sigue tu \begin{document})
\begin{document}

% --- PORTADA ---
\begin{titlepage}
    \maketitle
    \thispagestyle{empty} % Sin número de página en la portada
    
    \vspace{2cm}
    \begin{center}
        \textbf{\Large Información del Proyecto}\\[1cm]
        \begin{tabular}{ll}
            \toprule
            \textbf{Campo} & \textbf{Detalle} \\
            \midrule
            \textbf{Institución:} & Instituto Tecnológico Superior Oriente (ITSOEH) \\
            \textbf{Dominio:} & tics.edu.mx \\
            \textbf{Tema:} & Implementación de Redes IPv6 con Autoconfiguración y DHCP \\
            \textbf{Fecha:} & Noviembre 2025 \\
            \bottomrule
        \end{tabular}
    \end{center}
\end{titlepage}

% --- ÍNDICE ---
\newpage
\tableofcontents
\newpage

% --- CONTENIDO ---

% Capítulo 1: Introducción
\chapter{INTRODUCCIÓN}

Este reporte documenta la configuración de una infraestructura de red empresarial basada en protocolo IPv6, implementando dos modelos de asignación de direcciones: \textbf{DHCP Stateless (sin estado)} y \textbf{DHCP Stateful (con estado)}. La topología integra múltiples switches, routers y un backbone de conectividad redundante para garantizar alta disponibilidad y redundancia en la red.

La relevancia de este proyecto radica en la necesidad actual de las organizaciones de migrar hacia IPv6, dado el agotamiento del espacio de direcciones IPv4 y la creciente demanda de dispositivos conectados a Internet.

\chapter{OBJETIVOS DEL PROYECTO}

\begin{itemize}
    \item Diseñar e implementar una red IPv6 segura y escalable.
    \item Aplicar segmentación mediante VLANs para organización de tráfico.
    \item Implementar mecanismos de protección mediante Port Security.
    \item Configurar redundancia de routers mediante HSRP v2.
    \item Validar autoconfiguración de direcciones IPv6 en clientes.
    \item Documentar procedimientos de administración y seguridad.
\end{itemize}

\chapter{DESCRIPCIÓN GENERAL DE LA ARQUITECTURA}

\section{Topología de Red}

La infraestructura está compuesta por dos áreas principales de red, cada una con su propia estrategia de asignación de direcciones IPv6.

\subsection{Área Stateless (Zona de Autoconfiguración sin Estado)}

\subsubsection{Dispositivos}
\begin{itemize}
    \item \textbf{3 Switches de Acceso:} S1, S2, S3
    \item \textbf{2 Routers de Distribución:} R1, R2
    \item \textbf{1 Router de Borde (Core):} RA
    \item \textbf{Dominio IPv6:} \texttt{2001:db8:cafe::/48}
\end{itemize}

\subsubsection{Características}
\begin{itemize}
    \item Autoconfiguración de direcciones IPv6 mediante RA (Router Advertisements).
    \item DHCP Stateless para información adicional (DNS, dominio).
    \item Banda de administración: VLAN 55.
    \item Segmentación de usuarios y administrativos.
\end{itemize}

\subsection{Área Stateful (Zona de Configuración con Estado)}

\subsubsection{Dispositivos}
\begin{itemize}
    \item \textbf{3 Switches de Acceso:} S4, S5, S6
    \item \textbf{2 Routers de Distribución:} R3, R4
    \item \textbf{1 Router de Borde (Core):} RB
    \item \textbf{Dominio IPv6:} \texttt{2001:db8:3c4d::/48}
\end{itemize}

\subsubsection{Características}
\begin{itemize}
    \item Asignación completa de direcciones vía DHCP Stateful.
    \item Gestión centralizada de direcciones IPv6.
    \item Pools DHCP por VLAN.
    \item Redundancia mediante HSRP.
\end{itemize}

\section{Diagrama de Topología}

\begin{figure}[H]
    \centering
    \footnotesize % Reducir fuente para asegurar alineación
\begin{verbatim}
                        ┌─────────────────┐
                        │    Router RA    │
                        │ (Core Stateless)│
                        │  2001:db8:7:7:1 │
                        └────────┬────────┘
                                 │ g0/1
                    ┌────────────┴────────────┐
                    │                         │
              ┌─────▼────┐             ┌─────▼────┐
              │ Router R1│             │ Router R2│
              │ s0/0/1 RA│             │ s0/0/0 RA│
              └─────┬────┘             └─────┬────┘
                    │                        │
            ┌───────┼────────────────────────┴──────┐
            │       │                               │
       ┌────▼────┐ ┌▼────────┐ ┌────────┐      ┌────▼────┐
       │   S1    │ │   S2    │ │   S3   │      │   S4    │
       │Stateless│ │Stateless│ │Stateless│     │Stateful │
       └────┬────┘ └────┬────┘ └────┬───┘      └────┬────┘
            │           │           │               │
       (Usuarios)  (Usuarios)  (Usuarios)      (Usuarios)
\end{verbatim}
    \caption{Topología de Red IPv6 - Stateless y Stateful}
\end{figure}

\chapter{CONFIGURACIÓN POR ÁREA}

\section{ÁREA STATELESS - CONFIGURACIÓN DE ROUTERS}

\subsection{Router RA - Core/Borde}

\textbf{Rol:} Punto de interconexión entre las dos áreas de la red.

\subsubsection{Interfaces Configuradas}
\begin{itemize}
    \item \textbf{s0/0/1:} \texttt{2001:db8:1:1::1/64} (Enlace a R1)
    \item \textbf{s0/0/0:} \texttt{2001:db8:2:2::1/64} (Enlace a R2)
    \item \textbf{g0/1:} \texttt{2001:db8:7:7::1/64} (Enlace a RB)
\end{itemize}

\subsubsection{Rutas Estáticas}
\begin{itemize}
    \item Redes \texttt{2001:db8:cafe::/48} alcanzables vía R1 y R2.
    \item Ruta por defecto (\texttt{::/0}) hacia el exterior vía RB.
\end{itemize}

\subsection{Router R1 - Distribución Stateless (Activo)}

\textbf{Prioridad HSRP:} 150 (Activo)\\
\textbf{Dominio:} \texttt{2001:db8:cafe::/48}

\subsubsection{DHCP Pools Configurados}

\begin{lstlisting}[language=bash]
- DHCP-STATELESS-15: VLAN AdminStaff
- DHCP-STATELESS-45: VLAN Users
- DHCP-STATELESS-55: VLAN Administrative
- DHCP-STATELESS-65: VLAN Native
\end{lstlisting}

\subsubsection{Direcciones de Interfaz (Subinterfaces dot1Q)}

\begin{table}[H]
    \centering
    \begin{tabular}{@{}lc ll@{}} % @{} elimina espacio extra en bordes
        \toprule
        \textbf{Interfaz} & \textbf{VLAN} & \textbf{Dirección IPv6} & \textbf{Función} \\
        \midrule
        g0/1.15 & 15 & 2001:db8:cafe:15::1/64 & AdminStaff - Gateway Principal \\
        g0/1.45 & 45 & 2001:db8:cafe:45::1/64 & Users - Gateway Principal \\
        g0/1.55 & 55 & 2001:db8:cafe:55::1/64 & Admin - Gateway Principal \\
        g0/1.65 & 65 & 2001:db8:cafe:65::1/64 & Nativa - Gateway Principal \\
        \bottomrule
    \end{tabular}
    \caption{Direcciones IPv6 en Router R1}
\end{table}

\subsection{Router R2 - Distribución Stateless (Respaldo)}

\textbf{Prioridad HSRP:} 90 (Respaldo)\\
\textbf{Dominio:} \texttt{2001:db8:cafe::/48}

\begin{itemize}
    \item Misma configuración que R1 pero con prioridad 90.
    \item Enlace hacia RA vía s0/0/0 (redundancia).
    \item Actúa como respaldo automático en caso de fallo de R1.
\end{itemize}

\section{ÁREA STATELESS - CONFIGURACIÓN DE SWITCHES}

\subsection{Descripción General de VLANs (Stateless)}

\begin{table}[H]
    \centering
    \begin{tabular}{@{}ll l l@{}}
        \toprule
        \textbf{VLAN} & \textbf{Nombre} & \textbf{Propósito} & \textbf{Rango IPv6} \\
        \midrule
        15 & AdminStaff & Personal administrativo & 2001:db8:cafe:15::/64 \\
        45 & Users & Usuarios regulares & 2001:db8:cafe:45::/64 \\
        55 & Administrative & Gestión de dispositivos & 2001:db8:cafe:55::/64 \\
        65 & Native & Tráfico no etiquetado & 2001:db8:cafe:65::/64 \\
        \bottomrule
    \end{tabular}
    \caption{VLANs en Área Stateless}
\end{table}

\subsection{Switch S1}

\subsubsection{Configuración de Seguridad}

\begin{lstlisting}[language=bash]
hostname S1
enable password cisco
enable secret tics
banner motd # Solo acceso autorizado S1 #
service password-encryption
\end{lstlisting}

\subsubsection{Port-Security Implementada}

\begin{itemize}
    \item Interfaces f0/7-f0/8 (VLAN 15): Máximo 2 MAC + sticky + shutdown.
    \item Interfaces f0/9-f0/12 (VLAN 45): Máximo 2 MAC + sticky + shutdown.
    \item Interface f0/24 (VLAN 65): Máximo 2 MAC + sticky + shutdown.
\end{itemize}

\subsubsection{EtherChannel (Link Aggregation)}

\begin{lstlisting}[language=bash]
Port-Channel 1 (Activo - LACP):
- f0/1-f0/3 (channel-group 1 mode active)
- Configurado como trunk
- VLANs permitidas: 15,45,55,65
- VLAN nativa: 65

Port-Channel 2 (Activo - LACP):
- f0/4-f0/6 (channel-group 2 mode active)
- Configurado como trunk
- VLANs permitidas: 15,45,55,65
- VLAN nativa: 65
\end{lstlisting}

\subsection{Switches S2 y S3}

Configuración idéntica a S1 con las siguientes variaciones:

\begin{table}[H]
    \centering
    \begin{tabular}{ll}
        \toprule
        \textbf{Switch} & \textbf{VLAN de Gestión} \\
        \midrule
        S1 & 2001:db8:cafe:55::11/64 \\
        S2 & 2001:db8:cafe:55::12/64 \\
        S3 & 2001:db8:cafe:55::13/64 \\
        \bottomrule
    \end{tabular}
    \caption{IPs de Gestión - Switches Stateless}
\end{table}

\section{ÁREA STATEFUL - CONFIGURACIÓN DE ROUTERS}

\subsection{Router RB - Core/Borde}

\textbf{Rol:} Punto de interconexión del área Stateful hacia exterior.

\subsubsection{Interfaces Configuradas}
\begin{itemize}
    \item \textbf{s0/0/1:} \texttt{2001:db8:3:3::1/64} (Enlace a R3)
    \item \textbf{s0/0/0:} \texttt{2001:db8:4:4::1/64} (Enlace a R4)
    \item \textbf{g0/1:} \texttt{2001:db8:7:7::2/64} (Enlace a RA con redundancia)
\end{itemize}

\subsection{Router R3 - Distribución Stateful (Activo)}

\textbf{Prioridad HSRP:} 150 (Activo)\\
\textbf{Dominio:} \texttt{2001:db8:3c4d::/48}

\subsubsection{DHCP Pools Stateful}

\begin{lstlisting}[language=bash]
ipv6 dhcp pool DHCP-STATEFUL-10
  address prefix 2001:db8:3c4d:10::/64
  domain-name tics.edu.mx
exit

ipv6 dhcp pool DHCP-STATEFUL-20
  address prefix 2001:db8:3c4d:20::/64
  domain-name tics.edu.mx
exit

ipv6 dhcp pool DHCP-STATEFUL-30
  address prefix 2001:db8:3c4d:30::/64
  domain-name tics.edu.mx
exit

ipv6 dhcp pool DHCP-STATEFUL-40
  address prefix 2001:db8:3c4d:40::/64
  domain-name tics.edu.mx
exit
\end{lstlisting}

\subsubsection{Direcciones Gateway por VLAN}

\begin{table}[H]
    \centering
    \begin{tabular}{lc ll}
        \toprule
        \textbf{Interfaz} & \textbf{VLAN} & \textbf{Dirección IPv6} & \textbf{Método} \\
        \midrule
        g0/1.10 & 10 & 2001:db8:3c4d:10::1/64 & DHCP Stateful \\
        g0/1.20 & 20 & 2001:db8:3c4d:20::1/64 & DHCP Stateful (EUI-64) \\
        g0/1.30 & 30 & 2001:db8:3c4d:30::1/64 & DHCP Stateful (EUI-64) \\
        g0/1.40 & 40 & 2001:db8:3c4d:40::1/64 & DHCP Stateful (EUI-64) \\
        \bottomrule
    \end{tabular}
    \caption{Direcciones IPv6 en Router R3 (Stateful)}
\end{table}

\section{ÁREA STATEFUL - CONFIGURACIÓN DE SWITCHES}

\subsection{Descripción General de VLANs (Stateful)}

\begin{table}[H]
    \centering
    \begin{tabular}{ll ll}
        \toprule
        \textbf{VLAN} & \textbf{Nombre} & \textbf{Propósito} & \textbf{Rango IPv6} \\
        \midrule
        10 & AdminStaff & Personal administrativo & 2001:db8:3c4d:10::/64 \\
        20 & Users & Usuarios regulares & 2001:db8:3c4d:20::/64 \\
        30 & Administrative & Gestión de dispositivos & 2001:db8:3c4d:30::/64 \\
        40 & Native & Tráfico no etiquetado & 2001:db8:3c4d:40::/64 \\
        \bottomrule
    \end{tabular}
    \caption{VLANs en Área Stateful}
\end{table}

\chapter{COMPARATIVA: STATELESS vs STATEFUL}

\section{DHCP Stateless}

\subsection{Características}
\begin{itemize}
    \item Clientes generan su propia dirección IPv6 (autoconfiguración).
    \item DHCP solo proporciona información adicional (DNS, dominio).
    \item Flag ND: \texttt{other-config-flag} activado.
\end{itemize}

\subsection{Ventajas}
\begin{itemize}
    \item[\checkmark] Menor carga en servidor DHCP.
    \item[\checkmark] Configuración automática rápida.
    \item[\checkmark] Mayor escalabilidad.
    \item[\checkmark] Menor tráfico administrativo.
\end{itemize}

\subsection{Desventajas}
\begin{itemize}
    \item[\times] Menor control sobre asignaciones.
    \item[\times] Direcciones predecibles (basadas en MAC).
    \item[\times] Auditoría limitada de clientes.
\end{itemize}

\subsection{Caso de Uso}
Redes grandes con muchos clientes que requieren autoconfiguración rápida.

\section{DHCP Stateful}

\subsection{Características}
\begin{itemize}
    \item Servidor DHCP asigna toda la dirección IPv6.
    \item Control centralizado de direcciones.
    \item Flag ND: \texttt{managed-config-flag} activado.
\end{itemize}

\subsection{Ventajas}
\begin{itemize}
    \item[\checkmark] Control centralizado total.
    \item[\checkmark] Direcciones gestionadas.
    \item[\checkmark] Auditoría completa de clientes.
    \item[\checkmark] Pool de direcciones limitado.
    \item[\checkmark] Integración con DNS dinámico.
\end{itemize}

\subsection{Desventajas}
\begin{itemize}
    \item[\times] Mayor carga en servidor DHCP.
    \item[\times] Punto de fallo crítico.
    \item[\times] Mayor latencia de configuración.
    \item[\times] Requiere redundancia.
\end{itemize}

\subsection{Caso de Uso}
Redes corporativas que requieren auditoría y control absoluto.

\chapter{MECANISMOS DE REDUNDANCIA Y ALTA DISPONIBILIDAD}

\section{HSRP v2 (Hot Standby Router Protocol Version 2)}

\subsection{Implementación en Stateless}

\begin{lstlisting}[language=bash]
Grupo HSRP 15 (VLAN 15):
├─ R1: Priority 150 (Activo)
├─ R2: Priority 90 (Respaldo)
├─ Virtual IPv6: autoconfig
├─ Preempt: Enabled (R1 recupera control)
└─ Versión: 2

Grupos 45, 55, 65: Misma estructura
\end{lstlisting}

\subsection{Implementación en Stateful}

\begin{lstlisting}[language=bash]
Grupo HSRP 10 (VLAN 10):
├─ R3: Priority 150 (Activo)
├─ R4: Priority 90 (Respaldo)
├─ Virtual IPv6: autoconfig
├─ Preempt: Enabled
└─ Versión: 2

Grupos 20, 30, 40: Misma estructura
\end{lstlisting}

\subsection{Comportamiento en Failover}

\begin{enumerate}
    \item Si R1 falla $\Rightarrow$ R2 (priority 90) toma control en $<3$ segundos.
    \item R1 recuperándose $\Rightarrow$ toma control inmediatamente (preempt).
    \item Todos los clientes siguen comunicándose sin interrupción.
\end{enumerate}

\section{EtherChannel (Link Aggregation)}

\subsection{Implementación en Switches}

\begin{lstlisting}[language=bash]
Port-Channel 1: Agrupa f0/1-f0/3
├─ Protocolo: LACP (mode active)
├─ Velocidad efectiva: 300 Mbps (3x100 Mbps)
├─ Redundancia: Si falla 1 enlace, continúan 2
└─ Balanceo: Por dirección MAC de origen

Port-Channel 2: Agrupa f0/4-f0/6
├─ Mismo comportamiento que Port-Channel 1
└─ Distribuye tráfico hacia diferentes routers
\end{lstlisting}

\chapter{SEGURIDAD IMPLEMENTADA}

\section{Autenticación y Encriptación}

\subsection{En Todos los Dispositivos}

\begin{lstlisting}[language=bash]
enable password cisco           # Contraseña enable (texto plano)
enable secret tics              # Contraseña enable (encriptada)
service password-encryption     # Encripta contraseñas en config
banner motd # Solo acceso autorizado XX #
\end{lstlisting}

\subsection{SSH para Gestión Remota}

\begin{lstlisting}[language=bash]
username admin password admin       # Usuario local
ip domain-name itsoeh.edu           # Dominio requerido para RSA
crypto key generate rsa 1024        # Par de claves RSA
line vty 0 15
  transport input ssh               # Solo SSH (no Telnet)
  login local                        # Usa usuarios locales
\end{lstlisting}

\section{Port Security}

\subsection{Configuración Estándar}

\begin{lstlisting}[language=bash]
switchport port-security
switchport port-security maximum 2
switchport port-security mac-address sticky
switchport port-security violation shutdown
\end{lstlisting}

\subsection{Efectos}
\begin{itemize}
    \item Si MAC desconocida intenta acceder $\Rightarrow$ Puerto se desactiva.
    \item Protege contra cambios de hardware.
    \item Previene MAC spoofing y conexiones no autorizadas.
\end{itemize}

\section{Segmentación por VLAN}

\begin{itemize}
    \item Aislamiento de tráfico entre grupos.
    \item AdminStaff separado de Users.
    \item VLAN de administración separada.
    \item Implementación futura de ACLs por VLAN.
\end{itemize}

\chapter{ASIGNACIÓN DE DIRECCIONES IPv6}

\section{Arquitectura de Direccionamiento}

\textbf{Bloque Principal:} \texttt{2001:db8:0000::/32} (Documentación)

\subsection{División por Área}

\begin{lstlisting}[language=bash]
Área Stateless:       2001:db8:cafe::/48
├─ VLAN 15:           2001:db8:cafe:15::/64
├─ VLAN 45:           2001:db8:cafe:45::/64
├─ VLAN 55:           2001:db8:cafe:55::/64
└─ VLAN 65:           2001:db8:cafe:65::/64

Área Stateful:        2001:db8:3c4d::/48
├─ VLAN 10:           2001:db8:3c4d:10::/64
├─ VLAN 20:           2001:db8:3c4d:20::/64
├─ VLAN 30:           2001:db8:3c4d:30::/64
└─ VLAN 40:           2001:db8:3c4d:40::/64

Interconexión:
├─ RA-R1:             2001:db8:1:1::/64
├─ RA-R2:             2001:db8:2:2::/64
├─ RB-R3:             2001:db8:3:3::/64
├─ RB-R4:             2001:db8:4:4::/64
└─ RA-RB:             2001:db8:7:7::/64
\end{lstlisting}

\section{Esquema de Direccionamiento por Dispositivo}

\subsection{Área Stateless}

\begin{table}[H]
    \centering
    \begin{tabular}{lc ll}
        \toprule
        \textbf{Dispositivo} & \textbf{VLAN} & \textbf{Dirección} & \textbf{Función} \\
        \midrule
        R1 (Activo) & 15 & 2001:db8:cafe:15::1/64 & Gateway \\
        R2 (Respaldo) & 15 & 2001:db8:cafe:15::2/64 & Gateway Respaldo \\
        S1 Mgmt & 55 & 2001:db8:cafe:55::11/64 & Gestión Switch \\
        S2 Mgmt & 55 & 2001:db8:cafe:55::12/64 & Gestión Switch \\
        S3 Mgmt & 55 & 2001:db8:cafe:55::13/64 & Gestión Switch \\
        \bottomrule
    \end{tabular}
    \caption{Esquema de Direccionamiento - Stateless}
\end{table}

\subsection{Área Stateful}

\begin{table}[H]
    \centering
    \begin{tabular}{lc ll}
        \toprule
        \textbf{Dispositivo} & \textbf{VLAN} & \textbf{Dirección} & \textbf{Función} \\
        \midrule
        R3 (Activo) & 10 & 2001:db8:3c4d:10::1/64 & Gateway \\
        R4 (Respaldo) & 10 & 2001:db8:3c4d:10::2/64 & Gateway Respaldo \\
        S4 Mgmt & 30 & 2001:db8:3c4d:30::14/64 & Gestión Switch \\
        S5 Mgmt & 30 & 2001:db8:3c4d:30::15/64 & Gestión Switch \\
        S6 Mgmt & 30 & 2001:db8:3c4d:30::16/64 & Gestión Switch \\
        \bottomrule
    \end{tabular}
    \caption{Esquema de Direccionamiento - Stateful}
\end{table}

\chapter{FLUJO DE CONFIGURACIÓN DE CLIENTES}

\section{Proceso en Área Stateless}

\begin{enumerate}
    \item Cliente conecta a puerto de switch en VLAN 45 (Users)
    \begin{itemize}
        \item Puerto recibe tráfico sin etiqueta (Access).
    \end{itemize}
    
    \item Cliente envía Router Solicitation (RS)
    \begin{itemize}
        \item Busca información de red.
    \end{itemize}
    
    \item R1/R2 responden con Router Advertisement (RA)
    \begin{itemize}
        \item Prefix: \texttt{2001:db8:cafe:45::/64}
        \item Lifetime: Válido para autoconfiguración.
        \item \texttt{other-config-flag: TRUE} (DHCP Stateless disponible).
        \item Flag M/O: Managed = False, Other = True.
    \end{itemize}
    
    \item Cliente genera dirección IPv6
    \begin{itemize}
        \item Usa Prefix \texttt{2001:db8:cafe:45::/64}
        \item Interfaz Identifier = EUI-64 (basado en MAC).
        \item Resultado: \texttt{2001:db8:cafe:45::aaaa:bbbb:cccc} (ejemplo).
    \end{itemize}
    
    \item Cliente solicita información adicional vía DHCP
    \begin{itemize}
        \item Servidor: DHCP-STATELESS-45
        \item Información: DNS, dominio tics.edu.mx
        \item Confirmación recibida.
    \end{itemize}
    
    \item Cliente operativo con:
    \begin{itemize}
        \item Dirección IPv6: \texttt{2001:db8:cafe:45::aaaa:bbbb:cccc/64}
        \item Gateway: \texttt{2001:db8:cafe:45::1} (HSRP Virtual)
        \item DNS: Proporcionado por DHCP
        \item Dominio: tics.edu.mx
    \end{itemize}
\end{enumerate}

\textbf{Tiempo total:} $\sim 2$-$5$ segundos.

\section{Proceso en Área Stateful}

\begin{enumerate}
    \item Cliente conecta a puerto de switch en VLAN 20 (Users).
    
    \item Cliente envía Router Solicitation (RS).
    
    \item R3/R4 responden con Router Advertisement (RA)
    \begin{itemize}
        \item Prefix: \texttt{2001:db8:3c4d:20::/64}
        \item Lifetime: POCO VÁLIDO (indica reconfiguración).
        \item Flag M: Managed = TRUE $\Rightarrow$ DHCP obligatorio.
        \item Flag O: Other = FALSE
    \end{itemize}
    
    \item Cliente solicita dirección completa vía DHCPv6
    \begin{itemize}
        \item Servidor: DHCP-STATEFUL-20
        \item Pool: \texttt{2001:db8:3c4d:20::/64}
        \item Servidor asigna dirección única.
        \item Proporciona DNS y dominio.
        \item Lease: Configurable (ej: 1 día).
    \end{itemize}
    
    \item Cliente recibe asignación
    \begin{itemize}
        \item Dirección: \texttt{2001:db8:3c4d:20::xxxx} (asignada por servidor).
        \item Gateway: \texttt{2001:db8:3c4d:20::1} (HSRP Virtual).
        \item DNS: Del servidor DHCP.
        \item Lease: Límite de tiempo.
        \item MAC registrada en servidor.
    \end{itemize}
    
    \item Cliente operativo con control centralizado
    \begin{itemize}
        \item Auditoría completa disponible.
        \item Reconfigurable por administrador.
        \item Renovación periódica con servidor.
    \end{itemize}
\end{enumerate}

\textbf{Tiempo total:} $\sim 3$-$8$ segundos.

\chapter{ENRUTAMIENTO Y CONECTIVIDAD}

\section{Rutas Estáticas Configuradas}

\subsection{Router RA}

\begin{lstlisting}[language=bash]
ipv6 route 2001:db8:cafe::/48 s0/0/1 2001:db8:1:1::2
├─ Red Stateless vía R1

ipv6 route 2001:db8:cafe::/48 s0/0/0 2001:db8:2:2::2
├─ Red Stateless vía R2 (redundancia)

ipv6 route ::/0 2001:DB8:7:7::1
└─ Ruta por defecto hacia Internet vía RB
\end{lstlisting}

\subsection{Router RB}

\begin{lstlisting}[language=bash]
ipv6 route 2001:db8:3c4d::/48 s0/0/1 2001:db8:3:3::2
├─ Red Stateful vía R3

ipv6 route 2001:db8:3c4d::/48 s0/0/0 2001:db8:4:4::2
├─ Red Stateful vía R4 (redundancia)

ipv6 route ::/0 2001:DB8:7:7::2
└─ Ruta por defecto hacia Internet vía RA
\end{lstlisting}

\section{Conectividad Entre Áreas}

La conectividad entre áreas Stateless y Stateful se establece a través de los routers core RA y RB, permitiendo que usuarios de ambas áreas puedan comunicarse, siempre que no existan restricciones de firewall o ACLs.

\chapter{CONFIGURACIÓN DE SEGURIDAD RESUMIDA}

\section{Contraseñas y Credenciales}

\begin{table}[H]
    \centering
    \begin{tabular}{ll}
        \toprule
        \textbf{Tipo de Acceso} & \textbf{Credencial} \\
        \midrule
        Modo Enable (sin encriptar) & cisco \\
        Modo Enable (encriptado) & tics \\
        SSH (Usuario admin) & admin / admin \\
        VLAN de gestión Stateless & 55 \\
        VLAN de gestión Stateful & 30 \\
        \bottomrule
    \end{tabular}
    \caption{Credenciales del Sistema}
\end{table}

\section{Certificados y Encriptación}

RSA 1024-bit generado en cada dispositivo con dominio itsoeh.edu, usado para SSH y válido durante el ciclo de vida del dispositivo.

\section{Best Practices Implementadas}

\begin{itemize}
    \item[\checkmark] SSH habilitado (no Telnet).
    \item[\checkmark] Contraseñas encriptadas (enable secret).
    \item[\checkmark] Banner de advertencia legal.
    \item[\checkmark] Port Security en puertos de acceso.
    \item[\checkmark] Logging deshabilitado (sin logging console).
    \item[\checkmark] Autenticación local.
    \item[\checkmark] Encapsulación dot1Q en trunks.
\end{itemize}

\chapter{VENTAJAS DE LA ARQUITECTURA PROPUESTA}

\section{Alta Disponibilidad}
\begin{itemize}
    \item[\checkmark] Redundancia completa en todos los niveles.
    \item[\checkmark] HSRP v2 para failover automático.
    \item[\checkmark] EtherChannel para trunks sin punto único de fallo.
    \item[\checkmark] Múltiples rutas hacia cada destino.
\end{itemize}

\section{Escalabilidad}
\begin{itemize}
    \item[\checkmark] Arquitectura modular (Stateless + Stateful).
    \item[\checkmark] Fácil agregar nuevos switches/VLANs.
    \item[\checkmark] DHCP Stateless reduce carga de servidores.
    \item[\checkmark] Dirección IPv6 suficiente para crecimiento.
\end{itemize}

\section{Seguridad}
\begin{itemize}
    \item[\checkmark] Segmentación por VLAN.
    \item[\checkmark] Port Security en acceso.
    \item[\checkmark] SSH para gestión remota.
    \item[\checkmark] Encriptación de contraseñas.
    \item[\checkmark] Autenticación local.
\end{itemize}

\section{Administración}
\begin{itemize}
    \item[\checkmark] Configuración centralizada de DHCP.
    \item[\checkmark] HSRP simplifica gestión de gateways.
    \item[\checkmark] SSH facilita administración remota.
    \item[\checkmark] VLANs reducen broadcast.
\end{itemize}

\chapter{CONCLUSIONES}

Este proyecto implementa una infraestructura de red IPv6 empresarial robusta que demuestra:

\begin{enumerate}
    \item \textbf{Conocimiento de Protocolos IPv6:} Implementación completa de autoconfiguración y DHCP.
    \item \textbf{Redundancia:} HSRP v2 y EtherChannel para alta disponibilidad.
    \item \textbf{Seguridad:} Port Security, SSH, encriptación y segmentación.
    \item \textbf{Escalabilidad:} Diseño modular que permite crecimiento futuro.
    \item \textbf{Best Practices:} Configuración siguiendo estándares industriales.
\end{enumerate}

La arquitectura es adecuada para un ambiente académico/corporativo pequeño a mediano que requiera:
\begin{itemize}
    \item Autoconfiguración automática (Stateless).
    \item Control centralizado (Stateful).
    \item Redundancia completa.
    \item Gestión remota segura.
\end{itemize}

\chapter{RECOMENDACIONES FUTURAS}

\begin{enumerate}
    \item \textbf{Implementar Routing Dinámico:} Reemplazar rutas estáticas con OSPF v3.
    \item \textbf{Agregar Firewall:} Filtrar tráfico entre áreas con ACLs.
    \item \textbf{Monitoreo:} Implementar SNMP para monitoreo centralizado.
    \item \textbf{Backup:} Configurar respaldos automatizados de config.
    \item \textbf{Syslog:} Centralizar logs en servidor syslog.
    \item \textbf{QoS:} Priorizar tráfico de aplicaciones críticas.
    \item \textbf{DHCPv6 Redundancia:} Agregar servidor DHCP secundario.
    \item \textbf{VPN:} Conectar sitios remotos de forma segura.
\end{enumerate}

\newpage
\appendix

\chapter{Comandos Clave Utilizados}

\begin{lstlisting}[language=bash]
! Configuración básica
no logging console
hostname NOMBRE_DISPOSITIVO
enable password cisco
enable secret tics
banner motd # Solo acceso autorizado #
service password-encryption

! IPv6 Routing
ipv6 unicast-routing
ipv6 route DESTINO INTERFAZ NEXT-HOP

! HSRP v2
standby version 2
standby GROUP ipv6 autoconfig
standby GROUP priority PRIORIDAD
standby GROUP preempt

! DHCP
ipv6 dhcp pool NOMBRE-POOL
  address prefix PREFIJO
  domain-name DOMINIO.COM
exit

! VLAN y Trunking
vlan ID
  name NOMBRE
exit

interface INTERFAZ.ID
  encapsulation dot1Q ID
  no ip address
  ipv6 address DIRECCIÓN/PREFIJO
  ipv6 enable
exit

! EtherChannel
interface range INTERFAZ1 - INTERFAZN
  channel-group NUM mode active
  no shutdown
exit

interface port-channel NUM
  switchport mode trunk
  switchport trunk allowed vlan VLANS
  switchport trunk native vlan ID
exit

! SSH
username USUARIO password CONTRASEÑA
ip domain-name DOMINIO
crypto key generate rsa BITS
line vty 0 15
  transport input ssh
  login local
exit
\end{lstlisting}

\chapter{Comandos Show para Validación}

\section{Validación de Configuración General}

\begin{lstlisting}[language=bash]
show version                    # Versión del IOS
show running-config             # Configuración en RAM
show startup-config             # Configuración persistente
\end{lstlisting}

\section{Validación de Interfaces IPv6}

\begin{lstlisting}[language=bash]
show ipv6 interface brief       # Resumen de interfaces IPv6
show ipv6 interface g0/1.15     # Detalles de interfaz
show ipv6 interface s0/0/1      # Interfaz serial
\end{lstlisting}

\section{Validación de VLAN}

\begin{lstlisting}[language=bash]
show vlan brief                 # VLANs configuradas
show ipv6 interface vlan 55     # VLAN de gestión
\end{lstlisting}

\section{Validación de Redundancia}

\begin{lstlisting}[language=bash]
show standby brief              # Estado HSRP resumido
show standby 15 detail          # Detalles grupo HSRP 15
show etherchannel brief         # Estado EtherChannels
show etherchannel 1 detail      # Detalles Port-Channel 1
\end{lstlisting}

\section{Validación de DHCP}

\begin{lstlisting}[language=bash]
show ipv6 dhcp pool             # Pools DHCP configurados
show ipv6 dhcp binding          # Clientes DHCP
show ipv6 dhcp statistics       # Estadísticas DHCP
\end{lstlisting}

\section{Validación de Enrutamiento}

\begin{lstlisting}[language=bash]
show ipv6 route                 # Tabla de rutas IPv6
show ipv6 route static          # Rutas estáticas
show ipv6 neighbors             # Tabla de vecinos
\end{lstlisting}

\section{Validación de Seguridad}

\begin{lstlisting}[language=bash]
show port-security              # Resumen Port-Security
show port-security interface f0/7    # Port-Security por interfaz
show mac-address-table          # Tabla MAC completa
show mac-address-table secure   # MACs aprendidas por security
show ip ssh                     # Configuración SSH
\end{lstlisting}

\newpage

\begin{center}
    \vspace*{\fill}
    \Large\textbf{FIN DEL REPORTE}\\[1cm]
    \normalsize
    Instituto Tecnológico Superior Oriente del Estado de Hidalgo\\
    Configuración de Redes IPv6\\
    Noviembre 2025
    \vspace*{\fill}
\end{center}

\end{document}